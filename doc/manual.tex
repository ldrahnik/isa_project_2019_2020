\documentclass[a4paper,11pt]{article}

\usepackage[left=20mm, text={170mm, 240mm}, top=30mm]{geometry}
\usepackage[czech]{babel}
\usepackage[IL2]{fontenc}
\usepackage[utf8x]{inputenc}
\usepackage{enumitem}
\usepackage{times}
\usepackage{graphicx}
\usepackage[T1]{fontenc}
\usepackage{indentfirst}
\usepackage{amsfonts}
\usepackage{amsthm}
\usepackage{listings}

\setlength{\parskip}{1em}

\begin{document}

\begin{titlepage}

	\begin{center}

        \includegraphics[width=4.1cm,keepaspectratio,trim={1.2cm 1.2cm 1.2cm 1.2cm},clip]{./template-fig/VUT_symbol_barevne_CMYK_CZ}%symbol VUT

		{\Huge\textsc{Vysoké učení technické v~Brně}}\\
		\medskip
		{\huge\textsc{Fakulta informačních technologií}}\\
		\vspace{\stretch{0.382}}
		{\huge DNS resolver}\\
		\medskip
		{\LARGE ISA - Síťové aplikace a správa sítí}\\
		\vspace{\stretch{0.618}}
	\end{center}

    \noindent xdrahn00@stud.fit.vutbr.cz \Large {\hfill Brno, \today}

\end{titlepage}

\section{Úvod}

Cílem projektu bylo si přečíst RFC \cite{rfc10135}\cite{rfc13596} dokumenty a na základě nich implementovat DNS klienta za pomoci standartních síťových knihoven. Veškeré nejasné informace ze zadání bylo požadováno uvést do bloku Upřesnění zadání.

\section{Upřesnění zadání}

Program obsahuje nápovědu po zavolání parametru \textbf{-h}. Parametr server \textbf{-s} je nastavený jako povinný, v programu proto není nastavený defaultni server. Program obsahuje proti zadání možnost si zapnout debugovací režim \textbf{-d}. V zadání není přesně uvedené, že je kombinace reverzního typu dotazu PTR aktivovaného přepínačem \textbf{-x} a IPv6 typu dotazu \textbf{AAAA} aktivovaného přepínačem \textbf{-6} zakázaná, ale typ dotazu můžu být pouze jeden a proto není v programu tato kombinace povolená. Pokud je položen dotaz, na tento dotaz je přijata odpověď a program vypíše všechny výsledky (včetně prázdného seznamu výsledků), program vrací návratový kód \textbf{0}. Specifické návratové kódy jsou popsány v kategorii \textbf{Návratové kódy programu}. Program počítá s/kontroluje v odpovědích třídu internet \textbf{IN}.

\section{Program}

Program je členěný do modulů. Z funkce \textbf{main()} v hlavním modulu \textbf{dns.c} se jako první volá zpracování parametrů a jejich validace v modulu \textbf{params.c}. Při validaci se ověřuje zda není použita zakázaná kombinace parametrů jako například \textbf{-x} a \textbf{-6}, zda je zadán validní host případně validní IP adresa při reverzním dotazu. V tomto modulu také dochází ke konverzi IP adres na \textbf{nibble} formát při vyžádání reverzního dotazu.

Vstupní parametry byly naimplementovány následovně: rekurzivní dotaz je změna z 0 na 1 v hlavičce DNS dotazu, reverzní dotaz změnou typu z defaultního \textbf{A} na \textbf{PTR}, validaci IP IPv4/IPv6 adresy a její konverze na vyžadovaný \textbf{nibble} formát, IPv6 dotaz změnou typu z defaultního \textbf{A} na \textbf{AAAA}, port je změna v hlavičce DNS dotazu.

Poslání a přijetí odpovědi na samotný dotaz je vykonáno po zpracování argumentů v hlavním modulu \textbf{dns.c} v jeho hlavní metodě \textbf{dnsResolver()}. Modul dále obsahuje řadu podpůrných metod sloužící např. pro konverzi doménového jména na formát používaný v dotazu a zpět, čtení doménových jmen v odpovědi, konverze IP adres z binárního formátu do formátu pro výpis a zpět a výpis jednotlivých typů záznamů.

Program na zaslaný dotaz čeká 5 vteřin. Toto čekání je naimplementováno konfigurací socketu a nastavením timeoutu pro obdržení \textbf{SO\_RCVTIMEO}.

Program se v odpovědi postupně posouvá pomocí ukazatele a vypisuje pomocí podpůrných metod jednotlivé \textbf{RR's}. Pro rozumění samotné odpovědi jsou nadefinovány dle informací z RFC\cite{rfc10135} struktury pro DNS hlavičku (která se vyplňuje při odesílání dotazu, ale vrací se stejně tak s dotazy - v našem případě vždy s jedním dotazem), dále pro samotný dotaz (variabilní délka, typ dotazu a třída) a nejdůležitější struktura pro samotné dotazy, která obsahuje 2 variabilní položky název před přesně stanovenými daty typ, třída, ttl a délka dat následujících po této položce. 

\subsection{Formát výstupních dat}

Výstupní formát dat kopíruje formát uvedený v zadání k programu:

\begin{lstlisting}[frame=single, breaklines]
Authoritative: No/Yes, Recursive: No/Yes, Truncated: No/Yes
Question section (XY)
  [domenove jmeno], [typ zaznamu - viz. podporovane typy zaznamu], [trida - jina nez IN neni podporovana], [ttl - v sekundach], [ip adresa / domenove jmeno v pripade typu CNAME]
Answer section (XY)
Authority section (XY)
Additional section (XY)
\end{lstlisting}

\subsection{Podporované typy záznamů}

Program vypisuje všechny výsledky, které dostane a které jsou tohoto typu (zbylé započte do zobrazovaného počtu výsledků v dané sekci, ale nevypíše je):

\begin{enumerate}
\item A
\item AAAA
\item PTR
\item CNAME
\end{enumerate}

\subsection{Návratové kódy}

Pokud je položen dotaz, na tento dotaz je přijata odpověd’ a program vypíše všechny výsledky (včetně prázdného seznamu výsledků), program vrací návratový kód \textbf{0}. V jakkémkoliv jiném případě vrací následující návratové kódy:

\begin{enumerate}
\item nesprávné zavolání programu (chybí vyžadovaný parametr, uvedený parametr není validní, kombinace parametrů je zakázána, ...)
\item problém s alokováním prostoru
\item problém při navazování komunikace (konkrétně funkce socket())
\item problém při odesílání dotazu (konkrétně funkce sendto()) 
\item problém při přijímání odpovědi (konkrétně funkce recvfrom()) 
\item problém při zpracovávání přijaté odpovědí (nemá správný formát, je poškozená, ...)
\item problém při navazování komunikace (konkrétně funkce setsockopt())
\end{enumerate}

\section{Testování}

Byl vytvořen jednoduchý testovací server. Server odchytává pakety na určeném portu \textbf{-p} a jejich obsah vypisuje na stdout. Parametrem \textbf{-e} lze nastavit, aby server odchytnul pouze první packet, ten vypsal a pak se ukončil.

Testovací bash script spuštěný příkazem \textbf{sudo make test} spustí server na pozadí pro 1 příchozí packet a výstup / příchozí packet přesměruje do souboru, poté spustí program s konkrétním požadavkem. Pro porovnání byl použit nástroj \textbf{diff}.

\subsection{Debugovací režim}

Debugovací režim se aktivuje po uvedení option \textbf{-d}. Při zapnutém debugovacím režimu jsou vypisovány důležité informace sloužící především pro ladění programu, tedy pro případné odhalení chyby v programu, ale i pro větší pochopení jak program funguje. Tyto zprávy obsahují například konverze ze standartního zápisu doménového jména na zápis používaný v posílaném dotazu a naopak, konverze IP adres nebo přijatá data získaná z odpovědi před posouzením zda jsou správná. Veškeré zprávy směřují na standartní chybový výstup.

Tyto zprávy jsou posílányjako např. při převádění adres na formát použitý v DNS dotazu, 

\subsection{Valgrind}

Program byl otestován pomocí nástroje \textbf{Valgrind}. Veškerá paměť alokovaná v průběhu běhu programu je před jeho ukončením správně uvolněna. V programu nedochází k únikům paměti. K otestování byl použit Valgrind verze \textbf{3.13.0}. Valgrind byl spouštěn s parametry verbose \textbf{-v}, track origins \textbf{--track-origins=yes} a leak check \textbf{--leak-check=full}.

\subsection{Testování v referenčním prostředí}

Při překladu programu na serveru \textbf{eva.fit.vutbr.cz} bylo zjištěno, že nebyla nalezena nadefinovaná konstanta \textbf{IP\_MAXPACKET} z \textbf{netinet/ip.h}. Tato konstanta byla následně přidána.

\section{Ukázka spuštění}

Ukázka spustění s defaultním IPv4 typem dotazu `A`:

\begin{lstlisting}[frame=single,breaklines]
./dns -s 8.8.8.8 clients4.google.com
Authoritative: No, Recursive: No, Truncated: No

Question section (1):
  clients4.google.com, A, IN
Answer section (2):
  clients4.google.com, CNAME, IN, 0, clients.l.google.com
  clients.l.google.com, A, IN, 0, 216.58.201.110
Authority section (0):
Additional section (0):
\end{lstlisting}

\newpage

Ukázka spustění s nastaveným IPv6 typem dotazu \textbf{AAAA} aktivovaného použitím přepínače \textbf{-6}:

\begin{lstlisting}[frame=single,breaklines]
./dns -6 -s 147.229.8.12 www.fit.vutbr.cz
Authoritative: No, Recursive: No, Truncated: No

Question section (1):
  www.fit.vutbr.cz, AAAA, IN
Answer section (1):
  www.fit.vutbr.cz, AAAA, IN, 0, 2001:067C:1220:0809:0000:0000:93E5:0917
Authority section (4):
  fit.vutbr.cz, NS, IN, gate.feec.vutbr.cz
  fit.vutbr.cz, NS, IN, guta.fit.vutbr.cz
  fit.vutbr.cz, NS, IN, kazi.fit.vutbr.cz
  fit.vutbr.cz, NS, IN, rhino.cis.vutbr.cz
Additional section (3):
  guta.fit.vutbr.cz, A, IN, 0, 147.229.9.11
  kazi.fit.vutbr.cz, A, IN, 0, 147.229.8.12
  guta.fit.vutbr.cz, AAAA, IN, 0, 2001:067C:1220:0809:0000:0000:93E5:090B
\end{lstlisting}

Ukázka spuštění s nastaveným reverzním typem dotazu \textbf{PTR} akvivovaného přepínačem \textbf{-x} (je potřeba uvést validní IPv4/IPv6 adresu):

\begin{lstlisting}[frame=single,breaklines]
./dns -x -r -s 8.8.8.8 172.217.23.206
Authoritative: No, Recursive: Yes, Truncated: No

Question section (1):
  206.23.217.172.in-addr.arpa, PTR, IN
Answer section (2):
  206.23.217.172.in-addr.arpa, PTR, IN, 0, prg03s05-in-f206.1e100.net
  206.23.217.172.in-addr.arpa, PTR, IN, 0, prg03s05-in-f14.J
Authority section (0):
Additional section (0):
\end{lstlisting}

\newpage

\section{RPM balíček}

K vytvoření RPM balíčku byl do projektu přidán soubor .spec, který obsahuje informaci o verzi programu, kde program stáhnu (případně by bylo možné stažení zakombinovat do prep sekce), jak projekt nainstaluji, s jakými soubory program manipuluje atd. Sekce install odkazuje na soubor Makefile na příkaz `make install`. Aby byly cesty relativní, je souboru Makefile předána proměnná BUILD\_ROOT.

Před spuštěním vytváření RPM balíčku je potřeba, aby se archiv projektu `v0.1.tar.gz` nacházel ve složce `/home/<USER>/rpmbuild/SOURCES/`.

Na instalaci se doporučuje použít alien z důvodu ošetření závislostí a možnosti program odinstalovat. Nedoporučuje se instalovat RPM nebo dokonce instalovat program z Makefile:

  \lstset{language=Bash}
  \begin{lstlisting}[frame=single,breaklines]

  # USING MAKEFILE INSIDE PROJECT (THERE IS NO WAY HOW TO REMOVE, DO NOT DO IT)

  sudo make install


  # DIRECTLY FROM CREATED RPM

  # build rpm
  rpmbuild -ba <package-spec-file>.spec

  # install rpm
  rpm -Uvh <rpm-file>.rpm -v

  # find package
  rpm -qa | grep -i <package-name>

  # uinstall
  rpm -e <package-name>


  # USING ALIEN (HIGHLY RECOMMENDED)

  # build rpm
  rpmbuild -ba <package-spec-file>.spec

  # create .deb
  sudo alien <rpm-file>.rpm

  # install
  sudo dpkg -i <package-name>.deb

  # uinstall
  sudo dpkg --remove <package-name>

\end{lstlisting}

\section{Závěr}

Vývoj programu ověřil, že RFC(\cite{rfc10135}, \cite{rfc13596}) obsahují veškeré potřebné informace pro vývoj DNS klienta a že existuje spousta podpůrných hlavičkových souborů, které obsahují metody, které není potřeba znovu vytvářet a které usnadní práci při implementaci.

\newpage

\bibliographystyle{czechiso}

\bibliography{bibliography}

\end{document}
