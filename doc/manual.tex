\documentclass[a4paper,11pt]{article}

\usepackage[left=20mm, text={170mm, 240mm}, top=30mm]{geometry}
\usepackage[czech]{babel}
\usepackage[IL2]{fontenc}
\usepackage[utf8x]{inputenc}
\usepackage{enumitem}
\usepackage{scrextend}
\usepackage{lscape}
\usepackage{times}
\usepackage{graphicx}
\usepackage[T1]{fontenc}
\usepackage{lmodern}
\usepackage{indentfirst}
\usepackage{pgfplots}
\usepackage{pgfplotstable}
\usepackage{mathtools}
\usepackage{amsfonts}
\usepackage{amsthm}
\usepackage{booktabs}
\usepackage{longtable}

\setlength{\parskip}{1em}
\pgfplotsset{compat=1.15}


\begin{document}

\begin{titlepage}

	\begin{center}

        \includegraphics[width=4.1cm,keepaspectratio,trim={1.2cm 1.2cm 1.2cm 1.2cm},clip]{./template-fig/VUT_symbol_barevne_CMYK_CZ}%symbol VUT

		{\Huge\textsc{Vysoké učení technické v~Brně}}\\
		\medskip
		{\huge\textsc{Fakulta informačních technologií}}\\
		\vspace{\stretch{0.382}}
		{\huge DNS resolver}\\
		\medskip
		{\LARGE ISA - Síťové aplikace a správa sítí}\\
		\vspace{\stretch{0.618}}
	\end{center}

    \noindent xdrahn00@stud.fit.vutbr.cz \Large {\hfill Brno, \today}

\end{titlepage}

\section{Úvod}

Cílem projektu bylo si přečíst RFC dokumenty a na základě nich implementovat DNS klienta za pomoci standartních síťových knihoven. Veškeré nejasné informace ze zadání bylo požadováno uvést do bloku Upřesnění zadání.

\section{Upřesnění zadání}

Program obsahuje nápovědu po zavolání parametru `-h`. Parametr server `-s` je nastavený jako povinný, v programu proto není nastavený defaultni server. Program obsahuje proti zadání možnost si zapnout debugovací režim `-d`. V zadání není přesně uvedené, že je kombinace reverzního typu dotazu PTR aktivovaného přepínačem `-x` a IPv6 typu dotazu `AAAA` aktivovaného přepínačem `-6` zakázaná, ale typ dotazu můžu být pouze jeden a proto není v programu tato kombinace povolená. Pokud je položen dotaz, na tento dotaz je přijata odpověď a program vypíše všechny výsledky (včetně prázdného seznamu výsledků), program vrací návratový kód `0`. Specifické návratové kódy jsou popsány v kategorii `Návratové kódy programu`. Program počítá s/kontroluje v odpovědích třídu internet `IN`.

\section{Program}

Program je členěný do modulů. Z funkce `main()` v hlavním modulu `dns.c` se jako první volá zpracování parametrů a jejich validace v modulu `params.c`. Při validaci se ověřuje zda není použita zakázaná kombinace parametrů jako například `-x` a `-6`, zda je zadán validní host případně validní IP adresa při reverzním dotazu. V tomto modulu také dochází ke konverzi IP adres na `nibble` formát při vyžádání reverzního dotazu.

Vstupní parametry byly naimplementovány následovně: rekurzivní dotaz je změna z 0 na 1 v hlavičce DNS dotazu, reverzní dotaz změnou typu z defaultního `A` na `PTR`, validaci IP IPv4/IPv6 adresy a její konverze na vyžadovaný `nibble` formát, IPv6 dotaz změnou typu z defaultního `A` na `AAAA`, port je změna v hlavičce DNS dotazu.

Poslání a přijetí odpovědi na samotný dotaz je vykonáno po zpracování argumentů v hlavním modulu `dns.c` v jeho hlavní metodě `dnsResolver()`. Modul dále obsahuje řadu podpůrných metod sloužící např. pro konverzi doménového jména na formát používaný v dotazu a zpět, čtení doménových jmen v odpovědi, konverze IP adres z binárního formátu do formátu pro výpis a zpět a výpis jednotlivých typů záznamů.

Program na zaslaný dotaz čeká 5 vteřin. Toto čekání je naimplementováno konfigurací socketu a nastavením timeoutu pro obdržení `SO\_RCVTIMEO`.

Program se v odpovědi postupně posouvá pomocí ukazatele a vypisuje pomocí podpůrných metod jednotlivé `RR's`. Pro rozumění samotné odpovědi jsou nadefinovány dle informací z RFC struktury pro DNS hlavičku (která se vyplňuje při odesílání dotazu, ale vrací se stejně tak s dotazy - v našem případě vždy s jedním dotazem), dále pro samotný dotaz (variabilní délka, typ dotazu a třída) a nejdůležitější struktura pro samotné dotazy, která obsahuje 2 variabilní položky název před přesně stanovenými daty typ, třída, ttl a délka dat následujících po této položce. 

\subsection{Formát výstupních dat}

Výstupní formát dat kopíruje formát uvedený v zadání k programu:

```
Authoritative: No/Yes, Recursive: No/Yes, Truncated: No/Yes
Question section (XY)
  [doménové jméno], [typ záznamu - viz. podporované typy záznamů], [třída - jiná než IN není podporována], [ttl - v sekundách], [ip adresa / doménové jméno v případě typu CNAME]
Answer section (XY)
Authority section (XY)
Additional section (XY)
```

\subsection{Podporované typy záznamů}

Program vypisuje všechny výsledky, které dostane a které jsou tohoto typu (u zbylých zobrazí správný počet výsledků v dané sekci, ale nevypíše je):

```
A
AAAA
PTR
CNAME
```

\subsection{Návratové kódy}

```
1 - nesprávné zavolání programu (chybí vyžadovaný parametr, uvedený parametr není validní, kombinace parametrů je zakázána, ...)
2 - problém s alokováním prostoru
3 - problém při navazování komunikace (konkrétně funkce socket())
4 - problém při odesílání dotazu (konkrétně funkce sendto()) 
5 - problém při přijímání odpovědi (konkrétně funkce recvfrom()) 
6 - problém při zpracovávání přijaté odpovědí (nemá správný formát, je poškozená, ...)
7 - problém při navazování komunikace (konkrétně funkce setsockopt())
```

\section{Testování}

\subsection{Debugovací režim}

Debugovací režim se aktivuje po uvedení option `-d`. Při zapnutém debugovacím režimu jsou vypisovány důležité informace sloužící především pro ladění programu, tedy pro případné odhalení chyby v programu, ale i pro větší pochopení jak program funguje. Tyto zprávy obsahují například konverze ze standartního zápisu doménového jména na zápis používaný v posílaném dotazu a naopak, konverze IP adres nebo přijatá data získaná z odpovědi před posouzením zda jsou správná. Veškeré zprávy směřují na standartní chybový výstup.

Tyto zprávy jsou posílányjako např. při převádění adres na formát použitý v DNS dotazu, 

\subsection{Valgrind}

Program byl otestován pomocí nástroje `Valgrind`. Veškerá paměť alokovaná v průběhu běhu programu je před jeho ukončením správně uvolněna. V programu nedochází k únikům paměti. K otestování byl použit Valgrind verze `3.13.0`. Valgrind byl spouštěn s parametry verbose `-v`, track origins --track-origins=yes a leak check `--leak-check=full`.

test citate \cite{test_citate}

\subsection{Testování v referenčním prostředí}

TODO: překlad a teestování na eva/merlin

TODO: všechny 3 RFC uvést do citací a do úvodu do závorky a ještě ke strukturám.

test citate \cite{test_citate}

\section{Ukázka spuštění}

Ukázka spustění s defaultním IPv4 typem dotazu `A`:

```
./dns -s 8.8.8.8 clients4.google.com
Authoritative: No, Recursive: No, Truncated: No

Question section (1):
  clients4.google.com, A, IN
Answer section (2):
  clients4.google.com, CNAME, IN, 0, clients.l.google.com
  clients.l.google.com, A, IN, 0, 216.58.201.110
Authority section (0):
Additional section (0):
```

Ukázka spustění s nastaveným IPv6 typem dotazu `AAAA` aktivovaného použitím přepínače `-6`:

```
./dns -6 -s 147.229.8.12 www.fit.vutbr.cz
Authoritative: No, Recursive: No, Truncated: No

Question section (1):
  www.fit.vutbr.cz, AAAA, IN
Answer section (1):
  www.fit.vutbr.cz, AAAA, IN, 0, 2001:067C:1220:0809:0000:0000:93E5:0917
Authority section (4):
  fit.vutbr.cz, NS, IN, gate.feec.vutbr.cz
  fit.vutbr.cz, NS, IN, guta.fit.vutbr.cz
  fit.vutbr.cz, NS, IN, kazi.fit.vutbr.cz
  fit.vutbr.cz, NS, IN, rhino.cis.vutbr.cz
Additional section (3):
  guta.fit.vutbr.cz, A, IN, 0, 147.229.9.11
  kazi.fit.vutbr.cz, A, IN, 0, 147.229.8.12
  guta.fit.vutbr.cz, AAAA, IN, 0, 2001:067C:1220:0809:0000:0000:93E5:090B
```

Ukázka spuštění s nastaveným reverzním typem dotazu `PTR` akvivovaného přepínačem `-x` (je potřeba uvést validní IPv4/IPv6 adresu):

```
./dns -x -r -s 8.8.8.8 172.217.23.206
Authoritative: No, Recursive: Yes, Truncated: No

Question section (1):
  206.23.217.172.in-addr.arpa, PTR, IN
Answer section (2):
  206.23.217.172.in-addr.arpa, PTR, IN, 0, prg03s05-in-f206.1e100.net
  206.23.217.172.in-addr.arpa, PTR, IN, 0, prg03s05-in-f14.J
Authority section (0):
Additional section (0):
```

\section{Závěr}

Vývoj programu ověřil, že RFC obsahují veškeré potřebné informace pro vývoj DNS klienta a že existuje spousta podpůrných hlavičkových souborů, které obsahují metody, které není potřeba znovu vytvářet a které usnadní práci při implementaci.

\newpage

\bibliographystyle{czechiso}

\bibliography{bibliography}

\end{document}
